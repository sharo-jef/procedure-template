\documentclass[a4paper,papersize,12pt]{jsarticle}
\usepackage{./conf/jefprocedure}

% textlint-disable
\documentnumber{{\color{red}創}第{\color{red}0}号}
\author{\color{red}運営役員会長}
\title{\color{red}特定操縦技能審査実施要領}
\date{
    {
        \color{red}
        令和 1年 1月 1日 制  定 (創第0号)\\
        令和 2年 2月 1日 一部改定 (創第0号)
    }
}
% textlint-enable

\begin{document}
\maketitle

% textlint-disable-next-line ja-no-successive-word, no-doubled-conjunction
\section{総   則}
\subsection{目的}
本要項は、航空法(昭和27年法律第231号。以下「法」という。)第71条の3第1項の規定による
特定操縦技能の審査等に関して、法及び航空法施行規則(昭和27年運輸省令第56号。以下「規則」という。)
に規則される申請及び審査等を行うための方法等を定めることを目的とする。

\subsection{用語の定義}
この要領における用語の定義は、法及び規則に定めるもののほか、以下のとおりとする。

% textlint-disable
\begin{enumerate}[label=(\arabic*)]
    \item 「特定操縦技能審査」とは、法第71条の3に基づき国土交通大臣の認定を受けた操縦技能審査員が行う審査であって、国土交通大臣の行った技能証明を有する操縦者に対する審査であり、飛行前の2年以内に、操縦操作の能力、非常時の操作に関する知識、航空法規の改正点に関する知識等を有するかどうかについて確認することをいう。
    \item 「操縦技能審査員」とは、特定操縦技能の審査を行うのに必要な経験、知識及び能力を有することについて、法第71条の3に基づき国土交通大臣の認定を受けた者をいう。
    \item 「操縦技能審査員認定試験(以下「認定試験」という。)」とは、操縦技能審査員の認定を受けようとする者が、特定操縦技能審査を行うのに必要な基礎的知識及び能力を有することを確認するために、航空従事者試験官(以下「試験官」という。)が行う試験をいう。
    \item foo
    \item bar
    \begin{enumerate}[label=\aiu*.]
        \item hoge
        \item fuga
        \begin{enumerate}[label=(\aiu*)]
            \item a
            \item b
        \end{enumerate}
    \end{enumerate}
\end{enumerate}
% textlint-enable

% textlint-disable
\begin{fusoku}
\end{fusoku}
\begin{fusoku}[令和3年2月5日創第0号]
\end{fusoku}
\begin{fusokushou}
\end{fusokushou}
\begin{fusokushou}[令和3年2月5日創第0号]
\heading[施行期日]
\kou{本則は、平成24年4月1日から施行する。}
\heading[経過措置]
\kou{「相当認定」}

平成26年4月1日前においても、法第71条の3第1項の操縦技能審査員の認定に相当する
認定を行うことが出来る。
この認定を受けた者は、平成26年4月1日以降においては、操縦技能審査員の認定を受
けているとみなされる。
その基準については、2.1.「認定基準」、申請の方法等については、2.2.「認定の申
請」による。

なお、航空法施行規則の一部を改正する省令(平成24年国土交通省令第22号)附則第
6条第9項により「相当操縦技能審査員の証」(別記第2号様式)と引換えに「操縦技能
審査員の証」(規則第28号の6様式)の交付を希望申請する者は、「操縦技能審査員の証
引換 申請書」(別記様式)に関係書類を添えて、相当操縦技能審査員の認定を受けた
地方航空局へ提出することとする。

\sakujo{\kou}
\end{fusokushou}
% textlint-enable

\besshi[1]
\besshi[2]
\besshirimen[2]
\youshiki
\youshiki[1-1]
\youshiki[2]
\youshikirimen[2]
\end{document}
